% mscguide.tex
% v1.0, released 11 Nov 2019
% Copyright 2019 Cambridge University Press


\usepackage[english]{babel}
\usepackage[T1]{fontenc}
\usepackage[utf8]{inputenc}
\usepackage{amsmath}
\usepackage{utf8-symbols}
\usepackage{microtype}
\usepackage{nccmath}
\usepackage{textcomp}
\usepackage{mathtools}
\usepackage{xparse}
\usepackage{bbold}
\usepackage{stmaryrd}
\usepackage{wrapfig}
\usepackage{multirow}
\usepackage{fancyvrb}
\usepackage{xcolor}
\usepackage{float}
\usepackage{graphicx}
\usepackage{lineno}
\usepackage{dirtytalk}
\usepackage{epigraph}
\usepackage{tikz}
\usepackage{quiver}
\usepackage{minted}
\usepackage{amsmath}
\usepackage{amsthm}
\usepackage{amsfonts}
\usepackage{textcomp}
\usepackage{latexsym}
\allowdisplaybreaks

% correct spacing depending on the parameters.  In this case
% `all' exposes all the definitions of the package, and `british'
% makes sure that there is no comma after e.g. or i.e.
\usepackage[all,british]{foreign}

\newtheorem{example}[therm]{Example}
% \newtheorem{definition}[therm]{Definition}
% \newtheorem{lemma}[therm]{Lemma}

% \let\citep\cite
\usepackage{mathpartir}

% \usepackage[capitalize,noabbrev]{cleveref}
\usepackage{thmtools}

% The following let the reader click on the qed square
% to visit the Agda formalisation.
\newenvironment{linkproof}[2][]{%
    \begin{proof}[#1]
        \renewcommand{\qedsymbol}{\href{#2}{{ \(\qedsymbol\)}}}
    }{%
    \end{proof}
}
\newenvironment{linkdefinition}[2][]{%
    \def\linkagdadefinition{#2}
    \begin{definition}[#1]
}%
{%
    \hfill\href{\linkagdadefinition}{\(\diamond\)}
    \end{definition}
}


\numberwithin{equation}{section}
\usetikzlibrary{decorations.markings}
\usetikzlibrary{decorations.pathreplacing}
\usetikzlibrary{matrix}
\usetikzlibrary{arrows}
\usetikzlibrary{chains}
\usetikzlibrary{positioning}
\usetikzlibrary{scopes}
\usetikzlibrary{decorations.pathmorphing}
\graphicspath{{./images/}}
% MACROS

\newcommand{\proves}{\ensuremath{\vdash}}
\newcommand{\Mc}{\ensuremath{\mathcal{M}}}
\newcommand{\Bc}{\ensuremath{\mathcal{B}}}
\newcommand{\Cc}{\ensuremath{\mathcal{C}}}
\newcommand{\Rc}{\ensuremath{\mathcal{R}}}
\newcommand{\Vc}{\ensuremath{\mathcal{V}}}
\newcommand{\xpause}{\vspace*{-\baselineskip}\pause}
\newcommand{\mgu}{\mathrm{mgu}}
\newcommand{\dom}{\mathrm{dom}}
\newcommand{\nbe}{\mathrm{nbe}}
\newcommand{\FV}{\mathrm{FV}}
\newcommand{\Type}{\mathrm{Type}}
\newcommand{\To}{\Rightarrow}
\newcommand{\emptylist}{\langle\rangle}
\newcommand{\boxsort}{\square}
\newcommand{\da}{{\downarrow}}
\newcommand{\eval}[1]{\ensuremath{\llbracket#1\rrbracket}}
\newcommand{\valuation}[3]{\ensuremath{\llbracket#1\rrbracket_{#2}^{#3}}}
\newcommand{\transl}[1]{\ensuremath{\lceil#1\rceil}}
\newcommand{\id}{\mathtt{id}}
\newcommand{\Nbb}{\mathbb{N}}

\newcommand{\eps}{\varepsilon}
\newcommand{\JuvixCore}{\textsf{JuvixCore}}
\newcommand{\Geb}{\textsf{Geb}\,}
\newcommand{\Juvix}{\textsf{Juvix}}
\newcommand{\VampIR}{\textsf{VampIR}}
\newcommand{\LambdaIR}{\textsf{Lambda}}

\usepackage{cleveref}

\bibliographystyle{ACM-Reference-Format}
% \renewcommand{\baselinestretch}{2}


\begin{document}


% Title
\newcommand{\pubtitle}{Juvix to VampIR Pipeline}

\newcommand{\pubauthA}{Lukasz Czajka}
\newcommand{\pubaffilA}{a}
% \newcommand{\orcidA}{0000-0001-5477-1503}
\newcommand{\authemailA}{lukasz@heliax.dev}
% \newcommand{\eqcontribA}{}

% \newcommand{\pubauthB}{Poison Ivy}
% \newcommand{\pubaffilB}{a}
% \newcommand{\orcidB}{0000-0001-0000-0000}
% \newcommand{\authemailB}{poison@heliax.dev}
% \newcommand{\eqcontribB}{}

% \newcommand{\pubauthC}{Last Author}
% \newcommand{\pubaffilC}{a}
% \newcommand{\orcidC}{0000-0001-5477-1503}
% \newcommand{\authemailC}{mail@someinstitute.com}

% Institutions/Affiliations
\newcommand{\pubaddrA}{Heliax AG}

% Corresponding author mail
\newcommand{\pubemail}{\authemailA}

\newcommand{\pubabstract}{
This report explores two alternatives to Geb for Juvix-to-VampIR
compilation. The first alternative is a straightforward approach
based on full normalisation, which may be implemented relatively
quickly and used as a comparison baseline for all other
approaches. The second alternative is based on a pipeline of several
compiler transformations that together convert Juvix programs into
a form that can be directly translated to VampIR input.
}

% Description of the SI file, placed as a footnote
% \newcommand{\pubSI}{Electronic Supplementary Information (ESI) available:
% one PDF file with all referenced supporting information.}

% Any keywords to be displayed under the abstract
\keywords{ Juvix\sep 
Vamp-IR\sep 
Geb\sep 
compilation\sep 
normalisation\sep 
arithmetic circuits\sep
}

% Supplementary space between title/abstract and text, if needed
% \newcommand{\pubVadj}{0pt}

% ! DO NOT REMOVE OR MODIFY !
% Do not modify this file!

\title{\pubtitle}

\newcommand{\dg}{\textsuperscript{\textbf{\dag\textnormal{,}}}}

\ifdef{\pubauthA}{\author[\pubaffilA]{\pubauthA\ifdef{\orcidA}{~\protect\orcid{\orcidA}}{}\ifdef{\eqcontribA}{\dg}{}}}{}
\ifdef{\pubauthB}{\author[\pubaffilB]{\pubauthB\ifdef{\orcidB}{~\protect\orcid{\orcidB}}{}\ifdef{\eqcontribB}{\dg}{}}}{}
\ifdef{\pubauthC}{\author[\pubaffilC]{\pubauthC\ifdef{\orcidC}{~\protect\orcid{\orcidC}}{}\ifdef{\eqcontribC}{\dg}{}}}{}
\ifdef{\pubauthD}{\author[\pubaffilD]{\pubauthD\ifdef{\orcidD}{~\protect\orcid{\orcidD}}{}\ifdef{\eqcontribD}{\dg}{}}}{}
\ifdef{\pubauthE}{\author[\pubaffilE]{\pubauthE\ifdef{\orcidE}{~\protect\orcid{\orcidE}}{}\ifdef{\eqcontribE}{\dg}{}}}{}
\ifdef{\pubauthF}{\author[\pubaffilF]{\pubauthF\ifdef{\orcidF}{~\protect\orcid{\orcidF}}{}\ifdef{\eqcontribF}{\dg}{}}}{}
\ifdef{\pubauthG}{\author[\pubaffilG]{\pubauthG\ifdef{\orcidG}{~\protect\orcid{\orcidG}}{}\ifdef{\eqcontribG}{\dg}{}}}{}
\ifdef{\pubauthH}{\author[\pubaffilH]{\pubauthH\ifdef{\orcidH}{~\protect\orcid{\orcidH}}{}\ifdef{\eqcontribH}{\dg}{}}}{}
\ifdef{\pubauthI}{\author[\pubaffilI]{\pubauthI\ifdef{\orcidI}{~\protect\orcid{\orcidI}}{}\ifdef{\eqcontribI}{\dg}{}}}{}
\ifdef{\pubauthJ}{\author[\pubaffilJ]{\pubauthJ\ifdef{\orcidJ}{~\protect\orcid{\orcidJ}}{}\ifdef{\eqcontribJ}{\dg}{}}}{}
\ifdef{\pubauthK}{\author[\pubaffilK]{\pubauthK\ifdef{\orcidK}{~\protect\orcid{\orcidK}}{}\ifdef{\eqcontribK}{\dg}{}}}{}

\ifdef{\eqcontribA}{\equalcontrib{}}{}
\ifdef{\eqcontribB}{\equalcontrib{}}{}
\ifdef{\eqcontribC}{\equalcontrib{}}{}
\ifdef{\eqcontribD}{\equalcontrib{}}{}
\ifdef{\eqcontribE}{\equalcontrib{}}{}
\ifdef{\eqcontribF}{\equalcontrib{}}{}
\ifdef{\eqcontribG}{\equalcontrib{}}{}
\ifdef{\eqcontribH}{\equalcontrib{}}{}
\ifdef{\eqcontribI}{\equalcontrib{}}{}
\ifdef{\eqcontribJ}{\equalcontrib{}}{}
\ifdef{\eqcontribK}{\equalcontrib{}}{}

\ifdef{\pubaddrA}{\affil[a]{\pubaddrA}}{}
\ifdef{\pubaddrB}{\affil[b]{\pubaddrB}}{}
\ifdef{\pubaddrC}{\affil[c]{\pubaddrC}}{}
\ifdef{\pubaddrD}{\affil[d]{\pubaddrD}}{}
\ifdef{\pubaddrE}{\affil[e]{\pubaddrE}}{}
\ifdef{\pubaddrF}{\affil[f]{\pubaddrF}}{}
\ifdef{\pubaddrG}{\affil[g]{\pubaddrG}}{}
\ifdef{\pubaddrH}{\affil[h]{\pubaddrH}}{}

\contact{\pubemail}

%% Abstract
%% --------

\begin{abstract}
    \pubabstract{}
\end{abstract}

%% Keywords
%% --------

\ifdef{\pubkeywords}{\keywords{\pubkeywords}}{}

%% Adjusting vertical space
%% ------------------------

\ifdef{\pubVadj}{\verticaladjustment{\pubVadj}}{}

% The preprint DOI to be used as an link in the paper
\pubdoi{10.5281/zenodo.8268823}
\history{(Received April 25, 2023; Published: August 16, 2023; Version: \today)}

\begin{abstract}
  This report explores two alternatives to GEB for Juvix-to-VampIR
  compilation. The first alternative is a straightforward approach
  based on full normalization, which may be implemented relatively
  quickly and used as a comparison baseline for all other
  approaches. The second alternative is based on a pipeline of several
  compiler transformations which together convert Juvix programs into 
  a form that can be directly translated to VampIR input.
\end{abstract}

\begin{keywords}
Vamp-IR, GEB, normalization
\end{keywords}
\maketitle

\section{Introduction}\label{sec_introduction}

VampIR is a proof-system-agnostic language for writing arithmetic
circuits. Juvix targets VampIR as one of its backends, currently
through an external GEB-to-VampIR compiler. Juvix programs are first
translated into the JuvixCore language and then transformed in a
series of steps into a form which can be directly translated to GEB.

This report describes two possible alternative Juvix-to-VampIR
compilation approaches which do not depend on GEB.
\begin{enumerate}
\item Normalization approach (Section~\ref{sec_normalization}).

   Relatively easy to implement. Perhaps the most straightforward
   approach possible. Could serve as a baseline comparison for other
   approaches and the quickest way to get a working Juvix-to-VampIR
   compiler, but suffers from the (super-)exponential blow-up problem.
\item Transformation pipeline (Section~\ref{sec_pipeline}).

  Based on a pipeline of traditional compiler transformations which
  together convert JuvixCore programs into a form that can be directly
  translated to VampIR. A crucial transformation is that of
  defunctionalisation which removes higher-order functions (see
  Appendix~\ref{sec_defunctionalisation}). The pipeline is more robust
  than the normalization approach, but more time-consuming to
  implement. It is more efficient than unoptimized GEB, but less
  general and harder to extend.
\end{enumerate}

The conclusion of this report is the recommendation to implement the
normalization approach in order to have a first working version of
Juvix-to-VampIR compilation and a simple baseline compilation strategy
against which other approaches can be compared. The transformation
pipeline from Section~\ref{sec_pipeline} can be considered as an
``emergency'' alternative approach in case of significant unforeseen
issues with GEB in the future.

In the rest of this report, I first present a brief overview of the
features supported by Juvix, JuvixCore, GEB (current version) and
VampIR (Section~\ref{sec_features}). Then I discuss the fundamental
(super-)exponential blow-up problem for highly-branching programs and
its implications for any compilation strategy
(Section~\ref{sec_branching}). In Section~\ref{sec_geb}, I discuss the
issues with and limitations of the current implementation
of~GEB. Sections~\ref{sec_normalization} and~\ref{sec_pipeline}
describe the normalization approach and the proposed transformation
pipeline. Section~\ref{sec_comparison} presents a comparison of the
two approaches with GEB and each other. Section~\ref{sec_conclusion}
contains some recommendations together with their justifications. The
appendices contain technical details relevant to the normalization
approach (Appendix~\ref{sec_normal_forms}) and the transformation
pipeline (Appendix~\ref{sec_transformations}), and discussions of a
few more tangential technical issues (Appendix~\ref{sec_issues}).

{\bf Update:} As of Juvix version 0.3.5, the normalization approach
has been implemented. It exhibits the limitations predicted in this
report with regard to compiling highly-branching programs. The report
has been slightly updated to reflect the current state of the
normalization approach implementation and of GEB.

\section{Language features overview}\label{sec_features}

The following table provides an overview of language features
supported by Juvix, JuvixCore, current GEB and VampIR.

\vspace*{2mm}
\begin{minipage}[b]{\hsize}\centering
% \caption{Language features}
\begin{tabular}{lllll}
\hline
\textbf{Feature} & \textbf{Juvix} & \textbf{JuvixCore} & \textbf{GEB} & \textbf{VampIR} \\ \hline
General recursion (functions)   &  Yes   &  Yes   &  No  &   No  \\
First-class functions   &  Yes   &  Yes   &  Yes  &   No\footnote{VampIR higher-order and anonymous functions are not fully ``first-class'' because they cannot non-trivially interact with the field elements - they are essentially compilation-time-only.}  \\
Inductive data types  &  Yes   &  Yes   &  No  &   No  \\
Finite data structures  &  Yes   &  Yes   &  Yes\footnote{Encodable via products and coproducts.} & No\footnote{Pairs and lists in VampIR cannot non-trivially interact with the field elements. In general, finite JuvixCore data structures cannot be translated to them.}  \\
Prenex polymorphism  &  Yes   &  Yes   &  No  &  No\footnote{Polymorphic functions are supported, but not polymorphic data types which can interact non-trivially with field elements. Hence, monomorphisation is still necessary to translate from JuvixCore.} \\
Higher-rank polymorphism  &  Some   &  Yes   &  No  &   No  \\
Primitive integer type  &  Yes   &  Yes   &  No  &   Yes\footnote{VampIR ``integers'' are the field elements, not integers strictly speaking.} \\
Runtime error reporting  &  Yes   &  Yes   &  No  &   No \\
\hline
\end{tabular}
\end{minipage}

\section{Current state of the GEB implementation}\label{sec_geb}

Currently, GEB lacks the following features necessary for Juvix
integration.
\begin{itemize}
\item Primitive integers and arithmetic operations (critical).
\item Bounded algebraic datatype support (needed for recursive data structures).
\item Polymorphism (Juvix could alternatively implement monomorphisation).
\end{itemize}
These features are currently being worked on by the GEB team.

\section{The branching problem}\label{sec_branching}

Compiling highly-branching functional programs to circuits may result
in (super-)exponential circuit size blow-up. This branching problem is
a fundamental limitation on any direct non-interpretive compilation
strategy. The problem is that circuits directly encode all possible
computation paths.

With the proposed pipeline, Juvix programs would be translated to
VampIR programs with the increase in the program size only by a factor
polynomial in the unrolling depth limit. However, to compile the
generated programs to circuits, VampIR needs to essentially ``paste
in'' the bodies of functions at all call sites, which can lead to
circuits (super-)exponential in the size of the VampIR input
program. The recursive inlining performed by VampIR essentially
amounts to full normalization with the sharing of common normal
subexpressions of type integer.

The branching problem occurs in particular with recursive functions
which have more than one recursive function call occurrence in the
body. For example, the \texttt{filter} function

\begin{minted}{haskell}
  filter : (a -> Bool) -> List a -> List a
  filter p Nil = Nil
  filter p (Cons x xs) = if p x then Cons x (filter p xs) else filter p xs
\end{minted}
would be fully recursively ``unfolded'' up to the specified recursion
depth limit, with each recursive call unfolded separately, resulting
in circuit size exponential in the recursion depth limit. In this
example, the blow-up could be avoided by hoisting the common
subexpression \texttt{filter p xs}, i.e.
\begin{minted}{haskell}
  filter p (Cons x xs) =
    let xs' = filter p xs
    in if p x then Cons x xs' else xs'
\end{minted}
However, this is not always possible if there are two or more
recursive calls with essentially different arguments.

Fortunately, there is a large class of functional programs that can be
guaranteed to not cause an exponential blow-up (assuming the circuits
are DAGs not trees): those which use recursion linearly, i.e., for
which there is only one recursive call occurrence in each function
body. Note that folds over non-branching data structures (e.g.~lists)
fall into this category. In fact, the class of functions definable
with linear recursion is quite large (though it might be difficult or
inconvenient to reformulate programs into this format). All primitive
recursive functions on natural numbers are definable with first-order
linear recursion.

For first-order linearly recursive programs, the size of the final
circuit DAG should be at most proportional to the program size
multiplied by a factor polynomial in the recursion unrolling depth
limit, with the degree of the polynomial depending on the number of
functions in the program source.

For higher-order linearly recursive programs, to guarantee the lack of
blow-up one also needs to require that functions provided as arguments
to higher-order recursive functions are themselves not recursive.

With the normalization approach discussed in the next section, one
also needs to restrict recursion to natural numbers only.

\section{The normalization approach}\label{sec_normalization}

It is a simple lambda-calculus exercise to prove that any closed
JuvixCore normal-form lambda-term~$t$ of type $\mathrm{Int} \to \ldots
\to \mathrm{Int} \to \mathrm{Int}$ must have the form $t = \lambda x_1
\ldots x_n . b$ where $b$ is an expression built up from:
\begin{itemize}
\item the variables $x_1,\ldots,x_n$,
\item number literals,
\item arithmetic operations and comparisons,
\item if-then-else,
\item runtime errors.
\end{itemize}
See Appendix~\ref{sec_normal_forms} for a more detailed
analysis. After removing the error nodes, any lambda-term in such a form
can be directly translated to VampIR.

This suggests a straightforward compilation strategy with two main
steps:
\begin{enumerate}
\item Unrolling of recursive functions (already implemented).
\item Normalization (partly implemented).
\end{enumerate}
Once recursion is unrolled, all typed terms are strongly normalizing
and normalization can be performed without looping. After
normalization, runtime error nodes need to be removed, but this is low
effort (see Appendix~\ref{sec_errors}).

For normalization to work on anything but the tiniest toy programs one
needs to use the Normalization by Evaluation (NbE)
technique.\footnote{The difference between evaluation and
normalization is that evaluation does not reduce under binders, e.g.,
$\lambda x . (\lambda y . y) x$ is a value but not a normal
form. Evaluation is what mainstream functional programming languages
do (Haskell, OCaml, Lisp, \ldots). Normalization is what
dependently-typed proof assistants do (Coq, Agda, Idris, \ldots).} One
also needs to take care to propagate ``stuck'' runtime error
nodes\footnote{E.g.~reducing $(\mathrm{if}\,\mathrm{error}\,M\,N)$ to
$\mathrm{error}$.} and share common normal subexpressions of
type~$\mathrm{Int}$ using lets. Implementing NbE with these
modifications is more involved than a naive repetition of
$\beta\iota$-reductions. The implementation effort
is still much smaller than for the remaining parts of the pipeline
from Section~\ref{sec_pipeline}.

The problem with the normalization approach is that normalizing a
(simply-typed) lambda-term may increase its size
super-exponentially. In fact, the blow-up may be non-elementary, i.e.,
the size of the output may be greater than
\[
2^{2^{2^{\vdots^{2^n}}}}
\]
for any tower of~$2$s, where~$n$ is the size of the original
term.

Naive normalization without sharing of subexpressions of
type~$\mathrm{Int}$ may blow up the program size even for numeric first-order
linearly recursive programs. For example, normalising the $n$-times
composition of $\lambda x . f x x$ results in a normal form of size
exponential in~$n$. A remedy is to reduce $(\lambda x . t) s$ to
$\mathrm{let}\,x:=s\,\mathrm{in}\,t$ when~$s$ is a non-constant normal
form of type~$\mathrm{Int}$. After normalization, the lets need to be
hoisted (see Appendix~\ref{sec_let_hoisting}).

If normalization is implemented to share subexpressions of
type~$\mathrm{Int}$, it can be guaranteed that numeric first-order
linearly recursive programs do not cause an exponential
blow-up. Non-linearly recursive programs may still increase in size
super-exponentially.

Unfortunately, it is not possible to use the trick with lets for
recursion on non-numeric data types. First-order linearly recursive
programs recursing on data structures other than natural numbers or
integers may still cause a super-exponential blow-up.

However, if we ultimately target circuits, whenever this blow-up has
to happen, if it doesn't happen during Juvix-to-VampIR compilation, it
may happen when VampIR tries to generate the final circuit. VampIR
needs to essentially fully inline all functions, which amounts to full
normalization (with sharing of common normal subexpressions of
type~$\mathrm{Int}$). With the pipeline from
Section~\ref{sec_pipeline} the blow-up is delayed until VampIR
generates the circuit. With the normalization approach it already
happens when compiling Juvix to VampIR.

Nonetheless, it is a major disadvantage that the (super-)exponential
blow-up happens already when compiling to VampIR, which might make it
impossible in practice to generate any VampIR input at all. This would
prevent VampIR from optimizing the programs in the particular special
cases when the blow-up could be avoided. Also, if parts of the circuit
programs are to be ultimately interpreted in a VM, then the right
VM/interpreter code format is probably closer to GEB/VampIR than to
the initial JuvixCore without further transformations.

\section{The transformation pipeline}\label{sec_pipeline}

The proposed pipeline for compiling Juvix to VampIR directly, without
going through GEB, consists of several JuvixCore transformations which
together convert a JuvixCore program into a form that can be
straightforwardly translated into the input format of VampIR. The
transformations remove step-by-step the features of JuvixCore not
supported by VampIR (see Section~\ref{sec_features}).
\begin{enumerate}
\item Lambda-lifting (already implemented).
  {\small
  \begin{itemize}
  \item Removes anonymous lambda-abstractions by converting them into
    top-level named function definitions.
  \end{itemize} }
\item Monomorphisation (necessary for polymorphism).
  {\small
    \begin{itemize}
    \item Instantiates all type parameters in functions and datatypes with concrete types.
  \end{itemize} }
\item Defunctionalisation.
  {\small
    \begin{itemize}
    \item Removes higher-order functions.
  \end{itemize} }
\item Unrolling of recursive functions (already implemented).
  {\small
    \begin{itemize}
    \item Unrolls recursion up to some specified depth.
  \end{itemize} }
\item Hoisting of lets.
  {\small
    \begin{itemize}
    \item Moves lets upwards, out of applications.
  \end{itemize} }
\item Encoding of datatypes.
  {\small
    \begin{itemize}
    \item Encodes datatypes into tuples of numbers.
  \end{itemize} }
\item Encoding of runtime errors.
  {\small
    \begin{itemize}
    \item Encodes runtime errors with pairs.
  \end{itemize} }
\end{enumerate}
See Appendix~\ref{sec_transformations} for more detailed descriptions
of the transformations.

After performing the above transformations, we are left with a
collection of non-recursive first-order function definitions of the
form
\begin{minted}{ocaml}
  fun f x1 ... xn =
    let y1 = v1
        ...
        ym = vm
    in
    v
\end{minted}
where $x_1,\ldots,x_n,y_1,\ldots,y_m$ have the types of numbers or
tuples of numbers, and $v_1,\ldots,v_m,v$ are applicative
expressions built from:
\begin{itemize}
\item variables,
\item number literals,
\item fully applied functions,
\item arithmetic operations and comparisons,
\item tuple operations (tuple creation and destruction).
\end{itemize}
Any definition of this form can be directly translated to a VampIR
function definition. VampIR supports local definitions, arithmetic and
tuples (at circuit generation time).

\section{Comparison}\label{sec_comparison}

\subsection{Efficiency}

The proposed pipeline relies on traditional compiler transformations
to convert Juvix programs into a form that can be straightforwardly
translated to VampIR. The transformations are ``syntactic'' and
relatively ``low-level''. Such an approach might give better control
over the details of the generated VampIR code and make it easier to
generate efficient VampIR code (which can be compiled to
smaller/``faster'' circuits). The code generated with the proposed
transformation pipeline should be reasonably efficient without further
optimizations.

The normalization approach essentially ``fully inlines'' the program
by normalizing it. This may result in super-exponential blow-up at
compilation stage and thus failure to compile. But the code that does
compile without blow-up should be reasonably efficient.

GEB's efficiency improved since the first version of this report, but
it is still missing some features to allow for compilation from Juvix
and a thorough comparison.

\subsection{Generality}

GEB is more abstract and general, which in the long term might make it
more amenable to modular, structured extensions and to formal
verification. For example, according to Terence supporting higher-rank
polymorphism is not problematic. With the proposed pipeline, it is not
clear how to handle higher-rank polymorphism. Also, adding support for
dependent types may be more challenging with the proposed pipeline
than with~GEB.

\subsection{The branching problem}

The GEB approach, in the long run, might be better suited to address
the branching problem. Interpreting (parts of) the programs might
avoid circuit size blow-up, at the cost of significant interpretation
overhead. I'm not sure to what extent the current version of GEB
avoids the branching problem.

\section{Conclusion}\label{sec_conclusion}

I recommend the following.
\begin{enumerate}
\item Regardless of any other choices, implement the normalization
  approach from Section~\ref{sec_normalization} as a reference
  baseline.
  \medskip

  Justification:
  \begin{itemize}
  \item The normalization approach is perhaps the most straightforward
    compilation method possible for Juvix-to-VampIR. It can serve as a
    comparison baseline for more sophisticated methods.
  \item The normalization approach can be implemented relatively
    quickly -- should be doable till about the beginning of June.
  \item Implementing the normalization approach will give the GEB team
    more time to design the incorporation of the features needed by
    Juvix in a structured, modular manner.
  \end{itemize}
\item Continue the work on GEB. In case of significant unexpected
  difficulties with the GEB approach, implementing the pipeline from
  Section~\ref{sec_pipeline} could be considered.
  \medskip

  Justification:
  \begin{itemize}
  \item GEB is more general than the proposed pipeline, which in the
    long term may make it easier to extend it with higher-rank
    polymorphism, dependent types or other features. Potential
    formalization might also be easier.
  \item GEB might make it easier to solve the branching problem.
  \end{itemize}
\end{enumerate}


\clearpage
\appendix

\renewcommand{\thesection}{\Alph{section}}

\section{Analysis of normal forms}\label{sec_normal_forms}

We consider the problem of compiling a Juvix program with the main
function of type $\mathrm{Int} \to \mathrm{Int}$ into a one-argument
VampIR function~$f$. One would then use this function e.g.~in a VampIR
equation $f x = y$ with the input variables $x,y$.

A generalisation to multiple arguments of type $\mathrm{Int}$ is
straightforward. To handle first-order algeraic data types as the
arguments/result of the main function, one needs to specify how to
encode them into (tuples of) numbers.

Note that we restrict only the type of the main function. Other parts
of the program can use arbitrary types, be polymorphic, higher-order,
etc.

We consider an extended polymorphic lambda calculus without recursion
which essentially corresponds to JuvixCore after compilation of
pattern matching and the unrolling of recursion. For the sake of
brevity, we omit some features of JuvixCore (if-then-else, errors) but
the analysis extends to them.

\medskip

\noindent The types $\tau,\sigma$ are:
\begin{itemize}
\item primitive integer type $\mathrm{Int}$,
\item type variables $\alpha,\beta$,
\item parameterised inductive types $I \tau_1 \ldots \tau_n$,
\item function types $\tau \to \sigma$,
\item type quantification $\forall \alpha . \tau$.
\end{itemize}
The terms $t,s,r$ are:
\begin{itemize}
\item integer literals $n, m$,
\item arithmetic operations $t \odot s$ where $\odot \in \{ {+}, {-}, {*}, {/} \}$,
\item variables $x,y$,
\item application $t s$,
\item lambda-abstraction $\lambda x : \tau . t$,
\item type application $t \tau$,
\item type abstraction $\Lambda \alpha . t$,
\item inductive type constructors $c$,
\item case-expressions: $\mathrm{case}\,t\,\mathrm{of}\,\{ c_i x_1 \ldots x_{k_i} \To t_i \mid i = 1,\ldots,n\}$.
\end{itemize}
The reductions are:
\[
\begin{array}{c}
  (\lambda x : \tau . t) s \to_\beta t[x := s]\\
  (\Lambda \alpha . t) \tau \to_\beta t[\alpha := \tau]\\
  \mathrm{case}\,c_m s_1 \ldots s_{k_m}\,\mathrm{of}\,\{ c_i x_1 \ldots x_{k_i} \To t_i \mid i = 1,\ldots,n\} \to_\iota t_m[x_1 := s_1,\ldots,x_{k_m} := s_{k_m}]
\end{array}
\]
I omit the standard typing rules. The arithmetic operations
have type $\odot : \mathrm{Int} \to \mathrm{Int} \to \mathrm{Int}$.

\begin{lemma}
  Any closed (i.e.~without free variables) $\beta\iota$-normal term
  $t$ of type $\mathrm{Int} \to \mathrm{Int}$ must have the form $t =
  \lambda x : \mathrm{Int} . s$ where $s$ is a $\beta\iota$-normal
  term of type $\mathrm{Int}$ built up from:
  \begin{itemize}
  \item the variable $x$,
  \item integer literals $n, m$,
  \item arithmetic operations $\odot$.
  \end{itemize}
\end{lemma}

\begin{proof}
  This is an easy lambda-calculus exercise in normal form
  analysis. Because $t : \mathrm{Int} \to \mathrm{Int}$ is
  $\beta\iota$-normal and closed, it must be a lambda-abstraction
  $\lambda x : \mathrm{Int} . s$ with $s : \mathrm{Int}$ also
  $\beta\iota$-normal and $\FV(s) = \{x\}$. Then one proves by
  induction on the size of $\beta\iota$-normal $s : \mathrm{Int}$ with
  $\FV(s) = \{x\}$ that~$s$ has the required shape. The crucial
  observation is that if~$s$ is an application $s = h s_1 \ldots s_m$
  then $h$ cannot be a lambda-abstraction (because $s$ is
  $\beta$-normal), so it must be an arithmetic operation (because
  there are no other terms available having a function type with
  target~$\mathrm{Int}$). To exclude the possibility that~$s$ (or~$h$
  or~$t$ above) is a case-expression, one also needs to show by
  induction on $\beta\iota$-normal $r : I \tau_1 \ldots \tau_m$ with
  $\FV(r) = \{x\}$ that $r = c r_1 \ldots r_n$ for some constructor~$c$
  of~$I$.
\end{proof}

Any term in the form specified by the above lemma can be directly
translated into a VampIR function definition.

\clearpage
\section{Pipeline transformations}\label{sec_transformations}

In this appendix, I describe in more detail the transformations for
the pipeline proposed in Section~\ref{sec_pipeline}.

\subsection{Lambda-lifting}

\begin{itemize}
\item {\bf Necessity:} required
\item {\bf Workload:} already implemented
\end{itemize}

\noindent Lambda-lifting removes anonymous lambda-abstractions by converting
them into top-level named function definitions.

\medskip

\noindent For example,
\begin{minted}{haskell}
  f y z = map (\x -> x + y) z
\end{minted}
is transformed into
\begin{minted}{haskell}
  lam y x = x + y
  f y z = map (lam y) z
\end{minted}
This transformation has already been implemented for the WebAssembly pipeline.

\subsection{Monomorphisation}

\begin{itemize}
\item {\bf Necessity:} required for polymorphism
\item {\bf Workload:} medium
\end{itemize}

\noindent Monomorphisation instantiates all type parameters in
functions and datatypes with concrete types.

\medskip

\noindent For example,
\begin{minted}{haskell}
  data List a = Nil | Cons a (List a)

  map : (a -> b) -> List a -> List b
  map f Nil = Nil
  map f (Cons x xs) = Cons (f x) (map f xs)
\end{minted}
is transformed into
\begin{minted}{haskell}
  data List_Nat = Nil_Nat | Cons_Nat Nat List_Nat
  data List_String = Nil_String | Cons_String String List_String

  map_Nat_Nat : (Nat -> Nat) -> List_Nat -> List_Nat
  map_Nat_Nat f Nil_Nat = Nil_Nat
  map_Nat_Nat f (Cons_Nat x xs) = Cons_Nat (f x) (map_Nat_Nat f xs)

  map_String_Nat : (String -> Nat) -> List_String -> List_Nat
  map_String_Nat f Nil_String = Nil_Nat
  map_String_Nat f (Cons_String x xs) = Cons_Nat (f x) (map_String_Nat f xs)
\end{minted}
provided that \texttt{map} occurs in the program with $a = b =
\mathrm{Nat}$ and with $a = \mathrm{String}, b = \mathrm{Nat}$.

\medskip

Note that not all Juvix programs can be monomorphised, because Juvix
supports some higher-rank polymorphism. For example, the following is
a valid Juvix definition
\begin{minted}{agda}
f : {A : Type} -> ({B : Type} -> B -> B) -> A -> A;
f x := x x;
\end{minted}
It is not possible to (directly) monomorphise \texttt{f} because
$\lambda x . x x$ is not simply-typable.

With this approach, any program with non-monomorphisable higher-rank
polymorphism would cause a compilation error. One could also consider
a more general approach to monomorphisation, which would duplicate the
higher-rank polymorphic function arguments and move the type
quantifiers to the top, e.g., first changing the above \texttt{f} to
\begin{minted}{agda}
f : {A B C : Type} -> (B -> B) -> (C -> C) -> A -> A;
f x1 x2 := x1 x2;
\end{minted}
and adjusting all call sites appropriately. I'm not sure if this works
in full generality, but probably not.

From what Terence told me, higher-rank polymorphism is not a problem
for GEB, in principle. It is expected that GEB will support
higher-rank polymorphism in the future.

\subsection{Defunctionalisation}\label{sec_defunctionalisation}

\begin{itemize}
\item {\bf Necessity:} required
\item {\bf Workload:} medium
\end{itemize}

\noindent Defunctionalisation removes higher-order functions by
representing them as data structures and encoding unknown function
applications as case distinctions on this data structure.

\medskip

\noindent For example,
\begin{minted}{haskell}
  map : (Nat -> Nat) -> List Nat -> List Nat
  map f Nil = Nil
  map f (Cons x xs) = Cons (f x) (map f xs)

  z = map (g 0) (map f xs)
\end{minted}
is transformed into
\begin{minted}{haskell}
  data Fun_Nat_Nat = F | G Nat

  app_Nat_Nat : Fun_Nat_Nat -> Nat -> Nat
  app_Nat_Nat F x = f x
  app_Nat_Nat (G x) y = g x y

  map : Fun_Nat_Nat -> List Nat -> List Nat
  map f Nil = Nil
  map f (Cons x xs) = Cons (app_Nat_Nat f x) (map f xs)

  z = map (G 0) (map F xs)
\end{minted}
assuming that \texttt{f}, \texttt{g} are the only functions in the
entire program which occur as the head of a partial application of
type \mintinline{haskell}{Nat -> Nat}.

Note that, in general, one cannot use VampIR higher-order functions to
encode Juvix higher-order functions, because VampIR higher-order
functions are compile-time only and they cannot interact non-trivially
with the field elements. Thus they cannot be, e.g., directly stored in
a data structure encoded as a tuple of field elements.

In many cases, one could optimize by using VampIR higher-order
functions instead of defunctionalising, but not in general. In terms
of the effect on the final circuit, however, this seems equivalent to
doing the specialization optimization on the JuvixCore level
(i.e.~duplicating the higher-order functions with each known function
argument ``pasted in''). The downside of this optimization is that it
might result in program/circuit size blow-up (see also
Sections~\ref{sec_branching},\ref{sec_normalization}). It should
probably be performed only selectively.

\subsection{Unrolling of recursive functions}

\begin{itemize}
\item {\bf Necessity:} required for recursion
\item {\bf Workload:} already implemented
\end{itemize}

\noindent Unrolls recursion up to some specified depth.

\medskip

\noindent For example,
\begin{minted}{haskell}
  fact x = if x == 0 then 1 else x * fact (x - 1)
\end{minted}
is transformed into
\begin{minted}{haskell}
  fact0 x = error "recursion too deep"
  fact1 x = if x == 0 then 1 else x * fact0 (x - 1)
  fact2 x = if x == 0 then 1 else x * fact1 (x - 1)
  fact3 x = if x == 0 then 1 else x * fact2 (x - 1)
  ...
  fact{D} x = if x == 0 then 1 else x * fact{D-1} (x - 1)
  fact x = fact{D} x
\end{minted}
This transformation has already been implemented for the GEB pipeline.

\subsection{Hoisting of lets}\label{sec_let_hoisting}

\begin{itemize}
\item {\bf Necessity:} required
\item {\bf Workload:} low
\end{itemize}

\noindent Hoisting of lets moves lets upwards, out of applications and
other subexpressions.

\medskip

\noindent For example,
\begin{minted}{haskell}
  f (let x = let z = g a in z + 2 in x * b)
\end{minted}
is transformed into
\begin{minted}{haskell}
  let z = g a
  in
  let x = z + 2
  in
  f (x * b)
\end{minted}

\subsection{Encoding of datatypes}

\begin{itemize}
\item {\bf Necessity:} required
\item {\bf Workload:} medium
\end{itemize}

Elements of each datatype (i.e.~the constructors) need to be encoded
into (tuples of) numbers (field elements). Any unambiguous bit
encoding of constructors can be used. For example,
\begin{itemize}
\item the first 8 bits for the constructor tag (indicating which
  constructor),
\item the next 16 bits for the size of the encoded constructor,
\item the remaining bits for the encodings of constructor arguments in
  the order from left to right.
\end{itemize}
Note that after defunctionalisation all data types are first-order, so
every constructor argument is itself a constructor or a field
element. For finite bounded datatypes, the maximum number of bits
needed for the encoding can be calculated at compile time. If the
number of bits needed exceeds the number of bits available in a field
element, then we use an appropriate tuple of field elements. For
unbounded datatypes, one needs to set some bound on the depth of the
data structure and calculate the number of bits based on that. Pattern
matching on data types is then encoded by appropriate arithmetic and
tuple operations.

For finite datatypes which are not too deep, a simpler and more
readable encoding by nested tuples can be used. Let~$n$ be the maximum
number of constructor arguments in any inductive type in the entire
program. Then a constructor with~$k$ arguments is encoded by an
$n+1$-tuple where:
\begin{itemize}
\item the first element is the tag,
\item the next $k$ elements are the encodings of the constructor arguments,
\item the remaining $n - k$ elements are arbitrary elements of
  appropriate tuples (they're irrelevant since they'll never be
  accessed).
\end{itemize}
For the types to match, any integer constructor argument must also be
encodable as an $n+1$-tuple, e.g., by encoding it in the tag
field.

The nested-tuple encoding is a bit simpler but not very efficient with
datatypes of depth more than two or three.

\subsection{Encoding of runtime errors}\label{sec_errors}

\begin{itemize}
\item {\bf Necessity:} required
\item {\bf Workload:} low
\end{itemize}

\noindent Runtime errors need to be encoded, e.g., using pairs.

\medskip

\noindent For example,
\begin{minted}{haskell}
  f : Int -> Int
  f x = if x == 0 then error "Runtime error" else x - 1
\end{minted}
is transformed into
\begin{minted}{haskell}
  f : (Int, Int) -> (Int, Int)
  f (1, x) = (1, x)
  f (0, x) = if x == 0 then (1, 0) else (0, x - 1)
\end{minted}
where the first component of the pair indicates whether a runtime
error occurred.

\clearpage
\section{Other technical issues}\label{sec_issues}

\subsection{Recursion unrolling before defunctionalisation}

Unrolling recursion without defunctionalising first may result in
counterintuitive effective recursion depth limit. This is because with
higher-order functions one can ``cheat'' the recursion depth
limit. For instance, consider the following Juvix program:
\begin{minted}{agda}
iterate : {A : Type} → (A → A) → Nat → A → A;
iterate f zero x := x;
iterate f (suc n) x := f (iterate f n x);

plus : Nat → Nat → Nat;
plus := iterate suc;

mult : Nat → Nat → Nat;
mult m n := iterate (plus n) m 0;

exp : Nat → Nat → Nat;
exp m n := iterate (mult m) n 1;
\end{minted}
If one unrolls \texttt{iterate} up to depth~$k$ without
defunctionalising (i.e.~independently of its function argument), then
the effective recursion depth limit for the entire program (and the
size of the normal form) can be proportional to~$k^k$, because the
\texttt{exp} function iterates an iteration of iterations, each with
idependent depth limit~$k$. Note that the normal-form size blow-up may
happen even if \texttt{exp} is never actually expected to be called
with arguments that would result in exponential recursion depth.

After defunctionalising, the depth limits will depend on each other --
in \texttt{iterate} the calls via \texttt{app} to \texttt{plus} and
\texttt{mult} will ensure a decreasing depth limit for them.

\subsection{GEB encoding of higher-order functions}

As far as I understand, GEB essentially encodes higher-order functions
as field elements in a generic way (i.e., not depending on the
particular program but encoding the entire lambda-term generically)
and later essentially ``interprets'' them upon application (via the
\texttt{eval} morphism associated with the exponential object). The
difference with defunctionalisation seems to be that in
defunctionalisation we just do a simple switch to choose between a
finite number of functions that were present in the original program
and compiled ``normally'', while when the entire lambda-term is
encoded as a data structure all of it needs to be
``interpreted''. This might be several orders of magnitude less
efficient than defunctionalisation (depending on the notion of
``efficiency'' relevant to the circuit model). At this point, I am
unable to investigate this more thoroughly, because in its current
state GEB fails to compile any programs using higher-order functions.

This issue, however, will largely go away after GEB implements the
planned use of higher-order VampIR functions.

%\bibliography{ref.bib}

\end{document}
