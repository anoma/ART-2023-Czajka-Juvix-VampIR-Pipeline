

% Title
\newcommand{\pubtitle}{Juvix to VampIR Pipeline}

\newcommand{\pubauthA}{Lukasz Czajka}
\newcommand{\pubaffilA}{a}
% \newcommand{\orcidA}{0000-0001-5477-1503}
\newcommand{\authemailA}{lukasz@heliax.dev}
% \newcommand{\eqcontribA}{}

% \newcommand{\pubauthB}{Poison Ivy}
% \newcommand{\pubaffilB}{a}
% \newcommand{\orcidB}{0000-0001-0000-0000}
% \newcommand{\authemailB}{poison@heliax.dev}
% \newcommand{\eqcontribB}{}

% \newcommand{\pubauthC}{Last Author}
% \newcommand{\pubaffilC}{a}
% \newcommand{\orcidC}{0000-0001-5477-1503}
% \newcommand{\authemailC}{mail@someinstitute.com}

% Institutions/Affiliations
\newcommand{\pubaddrA}{Heliax AG}

% Corresponding author mail
\newcommand{\pubemail}{\authemailA}

\newcommand{\pubabstract}{
This report explores two alternatives to Geb for Juvix-to-VampIR
compilation. The first alternative is a straightforward approach
based on full normalisation, which may be implemented relatively
quickly and used as a comparison baseline for all other
approaches. The second alternative is based on a pipeline of several
compiler transformations that together convert Juvix programs into
a form that can be directly translated to VampIR input.
}

% Description of the SI file, placed as a footnote
% \newcommand{\pubSI}{Electronic Supplementary Information (ESI) available:
% one PDF file with all referenced supporting information.}

% Any keywords to be displayed under the abstract
\keywords{ Juvix\sep 
Vamp-IR\sep 
Geb\sep 
compilation\sep 
normalisation\sep 
arithmetic circuits\sep
}

% Supplementary space between title/abstract and text, if needed
% \newcommand{\pubVadj}{0pt}

% ! DO NOT REMOVE OR MODIFY !
\input{templates/ART/aux-preamble.tex}
% The preprint DOI to be used as an link in the paper
\pubdoi{10.5281/zenodo.8268274}
\history{(Received April 25, 2023; Published: August 16, 2023; Version: \today)}